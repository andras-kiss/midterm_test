\documentclass[a4paper, 11pt]{article}
\def\magyarOptions{hyphenation=huhyphn}
\usepackage{ae,aecompl}
\usepackage[T1]{fontenc}
\usepackage[utf8]{inputenc}
\usepackage[hungarian]{babel}
\usepackage{upgreek}
\usepackage{float}
\frenchspacing
\usepackage[dvips]{graphicx}
\usepackage[dvips]{color}
\usepackage{amsmath}
\usepackage{amssymb}
\usepackage{gensymb}
\usepackage{anysize}
\marginsize{1cm}{8cm}{0cm}{2cm}

\begin{document}
\textbf{Fizikai kémia példamegoldó szeminárium, 1. zh}

\textbf{2017. március 16.}
\thispagestyle{empty}

\begin{enumerate}

\item Egy 1 atm nyomású gázmintát összenyomunk kezdeti térfogata felére. Mekkora lesz a végső nyomás?

\item 25 \celsius$~$ hőmérsékleten izotermikusan és reverzibilisen összenyomunk 20$~$m$^3$ hélium gázt a kezdeti térfogat harmadára. A kezdeti nyomás 1$~$atm. Számolja ki a végső nyomást, a belsőenergia változást, ezen kívül $q$-t, $w$-t, és $\Delta H$-t! Tekintse a gázt tökéletes gáznak!

\item Számítsuk ki a $\Delta H$ és a $\Delta U$ közötti különbséget, ha 1 mol szürke ón (sűrűsége 5.75$~$g$\,$cm$^{-3}$) 10.0 bar nyomású fehér ónná (sűrűsége 7.31$~$g$\,$cm$^{-3}$) alakul. 298 K-en $\Delta H=~$2.1 kJ.

\item Számolja ki az entrópiaváltozását annak a folyamatnak, melynek során 100 J hőt ad át egy 300 K hőmérsékletű test egy 100 K hőmérsékletű testnek. A két testet tekintsük állandó hőmérsékletűnek. Ha hőerőgépet szeretnénk létrehozni e két test felhasználásával, mekkora lenne az elméletileg elérhető legjobb hatásfok?

\item Számolja ki a Daniell-cella reakció egyensúlyi állandóját! A két félcella reakció standard redukciós potenciálja $-0.760~$V és +0.34$~$V.

\end{enumerate}

\end{document}
