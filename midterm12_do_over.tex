\documentclass[a4paper, 11pt]{article}
\def\magyarOptions{hyphenation=huhyphn}
\usepackage{ae,aecompl}
\usepackage[T1]{fontenc}
\usepackage[utf8]{inputenc}
\usepackage[hungarian]{babel}
\usepackage{upgreek}
\usepackage{float}
\frenchspacing
\usepackage[dvips]{graphicx}
\usepackage[dvips]{color}
\usepackage{amsmath}
\usepackage{amssymb}
\usepackage{gensymb}
\usepackage[version=3]{mhchem}
\usepackage{anysize}
\marginsize{1cm}{8cm}{0cm}{2cm}

\begin{document}
\textbf{Fizikai kémia példamegoldó szeminárium, 1+2. (pót)zh.}\hfill \textbf{Név:}

\hfill \textbf{Csoport:}

\hfill \textbf{Dátum:}

\thispagestyle{empty}

\begin{enumerate}

\item Számítsuk ki a $\Delta H$ és a $\Delta U$ közötti különbséget, ha 1 mol szürke ón (sűrűsége 5.75$~$g$\,$cm$^{-3}$) 10.0 bar nyomású fehér ónná (sűrűsége 7.31$~$g$\,$cm$^{-3}$) alakul. 298 K-en $\Delta H=~$2.1 kJ.

\item Számolja ki az entrópiaváltozását annak a folyamatnak, melynek során 100 J hőt ad át egy 300 K hőmérsékletű test egy 100 K hőmérsékletű testnek. A két testet tekintsük állandó hőmérsékletűnek. Ha hőerőgépet szeretnénk létrehozni e két test felhasználásával, mekkora lenne az elméletileg elérhető legjobb hatásfok?

\item Számolja ki annak a galváncella reakciónak az egyensúlyi állandóját, melyben a két félcella reakció:

\begin{equation}
        \ce{Pb^{2+} + 2e^- <=> Pb(s)}
\end{equation}
\begin{equation}
        \ce{Zn^{2+} + 2e^- <=> Zn(s)}
\end{equation}

$E_1^\theta=-0.13~V$, $E_2^\theta=-0.76~V$. Melyik az anód és melyik a katód?

\item Egy elsőrendű reakció során a reakcióelegy térfogata a kezdeti $V_0\approx0~$L (elhanyagolható) értékről 5 perc alatt $30.2~$L$-$re növekszik. A reakció végén a térfogat 152.8$~$L. Számolja ki a sebességi állandót!

\item 25$~$g ismeretlen mintát oldunk CCl$_4$-ban. Az oldat forráspontja 77.4$~\celsius$, a tiszta CCl$_4$-é pedig 76.8$~\celsius$. A CCl$_4$ molális forráspontemelkedése 5.02 K$\,$kg$\,$mol$^{-1}$. Mekkora az ismeretlen anyag moláris tömege?

\end{enumerate}
\end{document}
