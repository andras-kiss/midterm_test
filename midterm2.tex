\documentclass[a4paper, 11pt]{article}
\def\magyarOptions{hyphenation=huhyphn}
\usepackage{ae,aecompl}
\usepackage[T1]{fontenc}
\usepackage[utf8]{inputenc}
\usepackage[hungarian]{babel}
\usepackage{upgreek}
\usepackage{float}
\frenchspacing
\usepackage[dvips]{graphicx}
\usepackage[dvips]{color}
\usepackage{amsmath}
\usepackage{amssymb}
\usepackage{gensymb}
\usepackage[version=3]{mhchem}
\usepackage{anysize}
\marginsize{1cm}{8cm}{0cm}{2cm}

\begin{document}
\textbf{Fizikai kémia példamegoldó szeminárium, 2. zh}\hfill \textbf{Név:}

\hfill \textbf{Csoport:}

\hfill \textbf{Dátum:}

\thispagestyle{empty}

\begin{enumerate}

\item Számolja ki annak a galváncella reakciónak az egyensúlyi állandóját, melyben a két félcella reakció:
 
\begin{equation}
        \ce{Pb^{2+} + 2e^- <=> Pb(s)}
\end{equation}
\begin{equation}
        \ce{Cu^{2+} + 2e^- <=> Cu(s)}
\end{equation}

$E_1^\theta=-0.13~V$, $E_2^\theta=+0.34~V$. Melyik az anód és melyik a katód? 

\item Egy elsőrendű reakció során a reakcióelegy abszorbanciája a kezdeti $A_0=0$ értékről 3 perc alatt $0.15-$re növekszik. A reakció végén az abszorbancia 1.1. Számolja ki a sebességi állandót!

\item Számolja ki a klórecetsav ($pK_a=2.86$) 10$^{-4}$ M vizes oldatában a klórecetsav disszociációfokát!  

\item Milyen tömegarányban állnak egymással a früktóz és a formaldehid azon mennyiségei, melyek azonos ozmotikus nyomást fejtenek ki? Válaszát indokolja! M$_{fr.}=180~$g$\,$mol$^{-1}$, M$_{fo.}=30~$g$\,$mol$^{-1}$.

\end{enumerate}
\end{document}
